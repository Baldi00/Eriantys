\documentclass{article}

\usepackage[italian]{babel}

\usepackage[letterpaper,top=2cm,bottom=2cm,left=3cm,right=3cm,marginparwidth=1.75cm]{geometry}

\usepackage{minted}
\usepackage[colorlinks=true, allcolors=blue]{hyperref}

\title{Protocollo di Comunicazione}
\author{Baldelli Andrea, Branchini Alessandro, Benatti Nicolas\\Gruppo AM27}

\begin{document}
\maketitle

\section{Struttura dei comandi inviati da Client e Server}

Ogni comando inviato sarà un JSON.\\
IL JSON ha la struttura seguente:

\begin{minted}{json}
{
	"command": "command_name",
	"param1": "param1",
	"param2": "param2"
}
\end{minted}

\noindent Il parametro "command" indica il tipo di comando eseguito e deve essere sempre presente.\\
Per aiutare la leggibilità il nome dei comandi è molto simile se non uguale ai metodi del Modello chiamati dal Controller.\\
Dopo command c'è una lista di parametri riferiti al comando.

\bigskip
\section{Connessione del Client al Server}

Il Client richiede all'utente indirizzo IP e porta del Server e vi si connette.\\
Se il Server non viene trovato il Client richiede nuovamente IP e porta all'utente.\\
Una volta connesso, il Server chiede al Client di inserire un nickname. Quest'ultimo lo chiede all'utente e lo invia al Server.\\

\noindent Server $\rightarrow$ Client

\begin{minted}{json}
{
	"command": "enterNickname",
}
\end{minted}

\noindent Client $\rightarrow$ Server

\begin{minted}{json}
{
	"command": "login",
	"nickname": "playerNickname"
}
\end{minted}

\noindent Se il nickname è già in uso il Server lo comunica al Client che lo richiede nuovamente all'utente, altrimenti comunica che il login è avvenuto con successo\\\\
\noindent Server $\rightarrow$ Client

\begin{minted}{json}
{
	"command": "nicknameAlreadyPresent"
}
\end{minted}

\noindent Server $\rightarrow$ Client

\begin{minted}{json}
{
	"command": "loginSuccessful"
}
\end{minted}

\bigskip
\section{Creazione/Unione alla partita}

Il Client comunica al Server la volontà di unirsi a una partita specificando numero di giocatori e se vuole partecipare a una partita in variante esperta.\\
Il Server inserisce il Client nella prima partita disponibile avente le caratteristiche richieste. Se non la trova ne crea una nuova con quelle caratteristiche.\\

\noindent Client $\rightarrow$ Server

\begin{minted}{json}
{
	"command": "joinMatch",
	"numPlayers": "numPlayers",
	"expertMatch": "true/false"
}
\end{minted}

\noindent Il Server a questo punto comunica al Client l'avvenuto inserimento in una partita\\

\noindent Server $\rightarrow$ Client

\begin{minted}{json}
{
	"command": "joinSuccessful"
}
\end{minted}

\bigskip
\section{Selezione mago e torre}

Quando sono arrivati tutti i giocatori a turno ogni giocatore sceglie torre e mago.\\
Il server manda al primo giocatore le possibili scelte di maghi/torri.\\

\noindent Server $\rightarrow$ Client

\begin{minted}{json}

{
	"command": "chooseWizardTower",
	"wizards": ["w1","w2", "..."],
	"towers": ["black", "white", "..."]
}
\end{minted}

\noindent Il Client una volta scelti mago e torre manda i dati al Server\\

\noindent Client $\rightarrow$ Server

\begin{minted}{json}
{
	"command": "addPlayer",
	"nickname": "playerNickname",
	"wizard": "wizardId",
	"tower": "towerColor"
}
\end{minted}

\bigskip
\noindent Quando il controller riceve i dati del giocatore chiama GameManager.addPlayer();


\bigskip
\section{Inizio partita}

Quando tutti i giocatori hanno finito di scegliere, il server inizializza la partita con il metodo preparation();\\
Dopo di ciò viene inviato un JSON a tutti i giocatori con lo stato iniziale della partita e un messaggio che dice che la partita è stata inizializzata (la libreria GSON fornisce dei metodi per la conversione in JSON di interi oggetti).\\

\noindent Server $\rightarrow$ All Clients
\begin{minted}{json}
{
	"command": "moveDone",
	"gameState": { "..." },
	"lastMove":
	    {
	        "command": "initialization"
	    }
}
\end{minted}

\bigskip
\section{Mosse che un giocatore può compiere}

\begin{itemize}
\item Giocare una carta assistente
\item Spostare studenti entrata $\rightarrow$ sala, entrata $\rightarrow$ isola
\item Muovere madre natura (specificando il numero di mosse)
\item Scegliere una tessera nuvola e prendere tutti gli studenti su di essa
\item Giocare una carta personaggio (specificando i parametri necessari)
\end{itemize}

\bigskip
\noindent Lista di comandi che svolgono queste mosse:

\begin{minted}{json}
{
	"command": "playerMovePlayAssistant",
	"assistant": "assistantId"
}

{
	"command": "playerMoveMoveStudentFromEntranceToHall",
	"studentColor": "studentColor"
}

{
	"command": "playerMoveMoveStudentFromEntranceToIsland",
	"studentColor": "studentColor",
	"islandId": "islandId"
}
{
	"command": "playerMoveMoveMotherNature",
	"steps": "steps"
}

\end{minted}
\newpage
\begin{minted}{json}
{
	"command": "playerMovePickStudentsFromCloud",
	"cloudId": "cloudId"
}

{
	"command": "playerMovePlayCharacter",
	"characterType": "characterType",
	"studentColor": "studentColor",
	"islandId": "islandId",
	"toExchangeFrom": ["studentColor1", "studentColor2", "..."],
	"toExchangeTo": ["studentColor1", "studentColor2", "..."],
	"toExchangeFromNumber": "numberOfExchanges",
	"toExchangeToNumber": "numberOfExchanges"
}
\end{minted}


\bigskip
\section{Risposta del Server a seguito di una mossa}

A seguito della ricezione di un comando mossa da parte di un Client, il Server modifica lo stato del gioco e invia lo stato modificato a tutti i Client in formato JSON insieme a un messaggio contenente l'ultima mossa fatta.\\

\noindent Server $\rightarrow$ All Clients
\begin{minted}{json}
{
	"command": "moveDone",
	"gameState": { "..." },
	"nickname": "playerOfTheMoveNickname",
	"lastMove":
	    {
	        "command": "...",
	        "param1": "...",
	        "param2": "..."
	    }
}
\end{minted}

\section{Gestione degli imprevisti}
\subsection{Logout dell'utente}\label{logout}
Siccome non gestiamo la resilienza alle disconnessioni, se un utente si disconnette volontariamente la partita deve terminare per tutti in qualunque fase di gioco.\\

\noindent Il Client manda un messaggio di logout al Server e il Server invia a tutti i Client un messaggio di chiusura forzata affinché si disconnettano.\\

\newpage

\noindent Client $\rightarrow$ Server
\begin{minted}{json}
{
	"command": "logout"
}
\end{minted}
\bigskip
\noindent Server $\rightarrow$ All Clients
\begin{minted}{json}
{
	"command": "forceEndMatch"
}
\end{minted}

\bigskip
\subsection{Server down}\label{serverdown}
Il Server, dopo che un Client si è connesso, inizia ad inviargli dei beat ad intervalli regolari.\\
Se il Client non rileva più l'arrivo di suddetti beat considera il Server disconnesso e si chiude.


\bigskip

\noindent Server $\rightarrow$ Client
\begin{minted}{json}
{
	"command": "beat"
}
\end{minted}

\subsection{Client down}
Allo stesso modo, ogni Client, dopo essersi connesso, inizia ad inviare dei beat al Server ad intervalli regolari.\\
Se il Server non rileva più l'arrivo dei suddetti beat da un Client lo considera disconnesso. A questo punto la partita termina e il Server invia a tutti gli altri Client connessi un messaggio di chiusura forzata affinché si disconnettano.

\bigskip

\noindent Client $\rightarrow$ Server
\begin{minted}{json}
{
	"command": "beat"
}
\end{minted}

\noindent Server $\rightarrow$ All Clients
\begin{minted}{json}
{
	"command": "forceEndMatch"
}
\end{minted}

\bigskip
\section{Fine della Partita}

La partita si è conclusa correttamente con un vincitore.\\
Il Server invia normalmente il messaggio di moveDone che contiene lo stato del gioco.\\
I Client leggendo il gameState si accorgono che è nello stato di GAME\_OVER. Sempre dal gameState reperiscono il vincitore e comunicano all'utente la fine della partita insieme con il vincitore.\\
A questo punto i Client si disconnettono.\\
Il Server si accorge della loro disconnessione ed esegue le operazioni di pulizia della partita.\\

\end{document}